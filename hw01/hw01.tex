\documentclass[12pt,a4paper]{article}
\usepackage[utf8]{inputenc}
\usepackage[czech]{babel}
\usepackage{graphicx}
\usepackage{fancyhdr}
\usepackage{amsmath}
\usepackage{float}
\usepackage[left=2cm,right=2cm,top=2cm,bottom=2cm]{geometry}
% Hyperlinks and PDF bookmarks
\usepackage[hidelinks,unicode,pdfencoding=auto]{hyperref}
\hypersetup{
	pdftitle={Domácí Úkol 01},
	pdfauthor={Jáchym Kouba},
	bookmarksopen=true,
	bookmarksnumbered=true
}

% Page numbering and header/footer setup
\pagestyle{fancy}
\fancyhf{}
\lhead{Jáchym Kouba}
\rhead{22. února 2026}
\cfoot{\thepage}

% Add dot after section numbers
\renewcommand{\thesection}{\arabic{section}.}
\renewcommand{\thesubsection}{\thesection\arabic{subsection}.}

\title{Domácí Úkol 01}
\author{Jáchym Kouba}
\date{22. února 2026}

\begin{document}

\maketitle
\thispagestyle{fancy}

\section{Úvod}

Úkol jsem zpracoval do sešitu a ofotil. Případné grafy, simulink modely, atd. jsem doplnil jako přílohy do 
pdf, jména kapitol odpovídají úkolu.

\section{Fotografie výpočtů}

\begin{figure}[H]
\centering
\includegraphics[width=0.8\textwidth]{hw1_1.jpg}
\caption{Výpočty - strana 1}
\label{fig:hw11}
\end{figure}

\begin{figure}[H]
\centering
\includegraphics[width=0.8\textwidth]{hw1_2.jpg}
\caption{Výpočty - strana 2}
\label{fig:hw12}
\end{figure}

\section{Úkol 01}

\subsection{Bod 3 - Lineární a Nelineární model}

\begin{figure}[H]
\centering
\includegraphics[width=0.48\textwidth]{task_01_3_linear.jpg}\hfill
\includegraphics[width=0.48\textwidth]{task_01_3_nonlinear.jpg}
\caption{Úkol 01, bod 3 - Lineární model (vlevo) a Nelineární model (vpravo)}
\label{fig:task01_3}
\end{figure}

\subsection{Bod 4 - Lineární a Nelineární model}

\begin{figure}[H]
\centering
\includegraphics[width=0.48\textwidth]{task_01_4_linear.jpg}\hfill
\includegraphics[width=0.48\textwidth]{task_01_4_nonlinear.jpg}
\caption{Úkol 01, bod 4 - Lineární model (vlevo) a Nelineární model (vpravo)}
\label{fig:task01_4}
\end{figure}

\subsection{Simulink model}

\begin{figure}[H]
\centering
\includegraphics[width=0.8\textwidth]{task_01_simulink.png}
\caption{Úkol 01 - Simulink model pro $v_0 = 0$}
\label{fig:task01_simulink}
\end{figure}

\section{Úkol 02}

\subsection{Bod 3 - Lineární a Nelineární model}

\begin{figure}[H]
\centering
\includegraphics[width=0.48\textwidth]{task_02_3_linear.jpg}\hfill
\includegraphics[width=0.48\textwidth]{task_02_3_nonlinear.jpg}
\caption{Úkol 02, bod 3 - Lineární model (vlevo) a Nelineární model (vpravo)}
\label{fig:task02_3}
\end{figure}

\subsection{Bod 4 - Lineární a Nelineární model}

\begin{figure}[H]
\centering
\includegraphics[width=0.48\textwidth]{task_02_4_linear.jpg}\hfill
\includegraphics[width=0.48\textwidth]{task_02_4_nonlinear.jpg}
\caption{Úkol 02, bod 4 - Lineární model (vlevo) a Nelineární model (vpravo)}
\label{fig:task02_4}
\end{figure}

\subsection{Simulink model}

\begin{figure}[H]
\centering
\includegraphics[width=0.8\textwidth]{task_02_simulink.png}
\caption{Úkol 02 - Simulink model pro $v_0 = 10$}
\label{fig:task02_simulink}
\end{figure}

\end{document}
